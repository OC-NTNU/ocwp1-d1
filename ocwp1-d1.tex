% Notes on writing:
% 
% * Use one sentence per line (no hard word wrapping), which makes merging different versions a lot easier.
% * Use \EM{}, \PO{} or \EA{} to add your personal comments or todos.
% * Using Git, the easiest way is probably to create your own branch and merge updates into the master branch. 


% From DoW (603773_DOW_2013-08-05[3].pdf, p.17):
%
% D1.3) Report on LBKD model and method in the climate change domain: 
% A survey of the key publications, approaches, software and resources on Literature-Based Knowledge Discovery will be conducted. 
% In addition, any work in this area specifically related to climate change will be surveyed. 
% This will result in survey paper on LBKD in the field of climate change [month 6]


\documentclass[11pt,twoside,a4paper]{report}
\usepackage{natbib}
\usepackage{graphicx}

\usepackage{todonotes}
% To hide all notes use 
% \usepackage[disable]{todonotes}


\newcommand{\EM}[1]{\todo[inline,author=EM,color=yellow]{#1}}
\newcommand{\PO}[1]{\todo[inline,author=PO,color=blue]{#1}}
\newcommand{\EA}[1]{\todo[inline,author=EA,color=green]{#1}}


\begin{document}

\title{Deliverable D1.1:\\ Literature-based Knowledge Discovery\\in Climate, Marine and Environmental Science}
\author{Erwin Marsi, Pinar \"Ozt\"urk, Elias Aamot}
\date{April 2014}
\maketitle

\abstract{}

\chapter{Introduction}


\chapter{Text Mining}

\todo[inline]{Describe text mining, especially entity detection, relation extraction and event extraction in BioNLP}

\todo[inline]{NLU, open IE}

\citet{Etzioni2011Search}

\chapter{Literature-based Knowledge Discovery}

\todo[inline]{key publications, approaches, software and resources on Literature-Based Knowledge Discovery}

\todo[inline]{inference and reasoning, relation to GOFAI}

\chapter{Text mining in Climate, Marine and Environmental Science}

\todo[inline]{Text mining work in related areas, e.g. Kyoto, GeoDeepDive, Text Mining for Marine Ecological Genomics}

\chapter{Ongoing and future work} 

\todo[inline]{Brief description of our current work and plans on LBD}

\bibliographystyle{plainnat}
\bibliography{ocwp1_d1}

\end{document}
